\documentclass[11pt,a4paper]{article}
\usepackage[margin=2.5cm]{geometry}
\usepackage[british]{babel}
\usepackage{amsmath}
\usepackage{amssymb}
\usepackage{hyperref}
\usepackage{parskip}

\title{Variational Bayes Implementation\\Summary of Changes to Original Algorithms}
\author{David Ewing}
\date{25 January 2026}

\begin{document}

\maketitle

\noindent
Dear Dr John,

I hope you're well. Here is an update on the VB project . I've adapted your original algorithms across the four model files.

\section*{Starting Point: M2-Hierarchical-Linear-Diagnostic.Rmd}

I began with your two core functions (normalmm.Gibbs and VB.mm) and wrapped them with diagnostic 
tools to fit the way I was making calls previously:

\begin{itemize}
    \item VB.mm\_with\_history() tracks $E[\tau_e]$, $E[\tau_u]$, and ELBO per iteration so we can monitor convergence
    \item Wrapper functions handle 3-chain Gibbs initialisation ($\tau_e=3/\tau_u=0.5$, $\tau_e=0.5/\tau_u=3$, $\tau_e=5/\tau_u=5$)
    \item Added convergence plots, ELBO trajectories, and 8-panel posterior comparisons
    \item The underlying algorithm is unchanged from your code
\end{itemize}

\section*{Reference Check: M0-Hierarchical-Diagnostic.Rmd}

I then removed these modified fuctions and replaced them with  your original functions with no diagnostic modifications, so I can demostrate the same as the code you wrote for me.
 M2 produces almost identical results, so this serves as a reference check:

\begin{itemize}
    \item Your exact normalmm.Gibbs() and VB.mm()
    \item Everything else is M2's diagnostic framework
    \item Same test conditions ($n=300$, $\tau_e=0.5$, $\tau_u=1.0$, $\beta=(0.5,-2,3)$)
\end{itemize}

\section*{Simplification: M1-Simple-Linear-Diagnostic.Rmd}

This removes the random effects to show how your algorithm reduces to simple linear regression:

\begin{itemize}
    \item No $u$ parameters or $Z$ matrix ($q=0$)
    \item Only $\tau_e$, no $\tau_u$
    \item normalmm.linear.Gibbs\_with\_history() samples $\beta \mid \tau_e$ and $\tau_e \mid \beta$
    \item Uses the same synthetic setup as the other models
\end{itemize}

\section*{Extension to Logistic: M3-Hierarchical-Logistic-Diagnostic.Rmd}

M3 extends M2 to binary responses, by

\begin{itemize}
    \item $y \sim \text{Bernoulli}(\text{logit}^{-1}(X\beta + Zu))$ instead of normal linear
    \item VB uses Laplace approximation (iterative reweighted least squares) rather than conjugate updates
    \item Gibbs uses Metropolis-Hastings for $(\beta,u) \mid \tau_u, y$ instead of conjugate sampling
    \item No $\tau_e$ since binary responses have no residual variance
    \item Same diagnostic framework but by totaly different algorithms
\end{itemize}

% (Removed standalone section per instructions; keep question within M3)

\section*{Consistent Test Conditions}

All files use the same setup:

\begin{itemize}
    \item $n = 300$, $p = 3$ fixed effects
    \item $\beta_{\text{true}} = (0.5, -2, 3)$
    \item $\tau_{e,\text{true}} = 0.5$ (where applicable); $\tau_{u,\text{true}} = 1.0$
    \item $Q$ scenarios: 5, 10, 20, 50, 100 groups
    \item Flat priors ($\alpha=0$, $\gamma=0$), seed 82171165
    \item Three-chain Gibbs initialisation following your pattern
\end{itemize}

\vspace{1em}

\noindent
All four files are now generating PDFs with diagnostic plots.  To demonstrate under-dispersion in VB for variance components by comparing against your Gibbs sampler across different group sizes and model types.

Can you review and advise, particularly on how to handle the M3 algorithmic difference.

\vspace{2em}

\noindent
Best regards,

\noindent
David Ewing

\end{document}
