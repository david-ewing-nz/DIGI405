\documentclass[11pt]{article}

% ---- Packages (minimal, traditional) ----
\usepackage{geometry}
\usepackage{graphicx}
\usepackage{amsmath, amssymb}
\usepackage{booktabs}
\usepackage{hyperref}

\geometry{margin=1in}

% ---- Title information ----
\title{Title of the Technical Paper}
\author{Author Name \\ Affiliation}
\date{\today}

\begin{document}

\maketitle

% ---- Abstract ----
\begin{abstract}
Brief summary of the problem, the approach taken, the main results,
and the key conclusion. Typically 150--250 words.
\end{abstract}

% ---- Keywords (optional) ----
\noindent\textbf{Keywords:} keyword1, keyword2, keyword3

% ---- Main sections ----
\section{Introduction}
Introduce the problem and its importance.
Provide context and motivation.
Identify the gap in existing work.
Summarise your approach and list the main contributions.
End with a brief outline of the paper.

\section{Related Work}
Review relevant prior work.
Highlight limitations or gaps.
Position your work relative to existing approaches.

% Sometimes merged with Introduction in shorter papers

\section{Problem Definition}
Formally define the problem.
State assumptions, constraints, and notation.
Clarify objectives and evaluation criteria.

\section{Methods}
Describe the proposed approach in detail.
Include:
\begin{itemize}
  \item Model or system design
  \item Algorithms or procedures
  \item Data sources and preprocessing
  \item Implementation details needed for reproducibility
\end{itemize}

\section{Experimental Setup}
Describe how the method is evaluated.
Include:
\begin{itemize}
  \item Datasets and splits
  \item Baselines
  \item Evaluation metrics
  \item Hardware and software environment
\end{itemize}

\section{Results}
Present the experimental results.
Use tables and figures where appropriate.
Compare against baselines and highlight key findings.

\section{Discussion}
Interpret the results.
Explain why they occur.
Discuss limitations, trade-offs, and failure cases.
Address threats to validity.

\section{Conclusion}
Summarise the work and main findings.
State the practical or theoretical impact.
Outline directions for future work.

% ---- Back matter ----
\section*{Acknowledgements}
Optional acknowledgements.

\bibliographystyle{plain}
\bibliography{references}

% ---- Appendix (optional) ----
\appendix
\section{Additional Material}
Supplementary proofs, derivations, extended results, or implementation details.

\end{document}
