% Laplace Approximations and Hybrid Methods: Future Directions
% Comparison and Recommendations Table

\section*{Future Directions: Hybrid Methods and Validation}

\subsection*{Hybrid Methods: VI for Speed + Gibbs for Certainty}

Hybrid methods that use VI for fast exploration and Gibbs sampling for final refinement offer a practical middle ground. This approach leverages the computational advantages of VI whilst maintaining the reliability of sampling-based methods.

\subsection*{Comparison of Current and Proposed Approaches}

\begin{center}
\begin{tabular}{l p{3cm} p{3cm} p{3cm} l}
\toprule
\textbf{Method} & \textbf{Status} & \textbf{Accuracy} & \textbf{Speed} & \textbf{Practical Use} \\
\midrule
Mean-Field VI & \checkmark Implemented & Moderate (under-dispersed) & Very Fast & ✓ Fixed effects estimation \\
\midrule
Gibbs Sampling & \checkmark Implemented & Excellent (gold standard) & Slow ($O(Q^2)$ scaling) & ✓ Reference baseline \\
\midrule
Hybrid: VI + Gibbs & ⧖ Proposed & Excellent (combines strengths) & Moderate & ✓ Production systems \\
\midrule
Stan (NUTS/HMC) & ⧖ Proposed & Excellent (gradient-based) & Moderate & ✓ Independent validation \\
\bottomrule
\end{tabular}
\end{center}

\subsection*{Hybrid Method Details}

\subsection*{Hybrid Method: VI Exploration + Gibbs Refinement}

\begin{enumerate}
\item \textbf{Phase 1 (VI Exploration):} Run mean-field VI for fast initial estimate
  \begin{itemize}
    \item Provides point estimate and approximation quality assessment via SD ratios
    \item Identifies parameters with severe under-dispersion in $O(0.1\text{s})$
  \end{itemize}

\item \textbf{Phase 2 (Gibbs Refinement):} Use VI results to initialise Gibbs sampler
  \begin{itemize}
    \item Warm-start from VI posterior mean instead of prior
    \item Reduces burn-in iterations required (potential 20–30\% savings)
    \item Final uncertainty estimates are reliable and unbiased
  \end{itemize}

\item \textbf{Quality Assurance:} Compare VI and Gibbs estimates
  \begin{itemize}
    \item Compute SD ratios to assess VI approximation quality
    \item Flag parameters with severe under-dispersion
    \item Determine whether correction factors are needed (e.g., 1.4–2.5 for variance components)
  \end{itemize}
\end{enumerate}

\subsection*{Stan (NUTS/HMC) for Independent Validation}

Stan's No-U-Turn Sampler (NUTS) provides gradient-based Hamiltonian sampling, offering:
\begin{itemize}
\item Independent algorithmic implementation (reduces risk of shared bugs)
\item Automatic differentiation for complex models
\item Established convergence diagnostics (Rhat, effective sample size)
\item Validation pathway: confirm that VI under-dispersion findings hold across multiple MCMC implementations
\end{itemize}

\subsection*{Recommendation Matrix}

\begin{center}
\begin{tabular}{l l l l}
\toprule
\textbf{Scenario} & \textbf{Primary Concern} & \textbf{Recommended Method} & \textbf{Rationale} \\
\midrule
\multirow{2}{*}{Fixed effects only} & Speed + acceptable accuracy & Mean-Field VI & 128× faster than Gibbs \\
& Moderate Q & & SD ratios 0.90–0.95 \\
\midrule
\multirow{2}{*}{Variance estimation} & Accurate hyper-parameters & Gibbs or NUTS & VI under-dispersed by 48\% \\
& Posterior credibility essential & & Reliable SD ratios $\geq 0.85$ \\
\midrule
\multirow{2}{*}{Production system} & Speed + reliability & Hybrid (VI + Gibbs) & Combines fast exploration \\
& Real-time constraints & & with sampling-based certainty \\
\midrule
\multirow{2}{*}{Publication/validation} & Independent confirmation & Stan (NUTS) & Gradient-based algorithm \\
& Methods comparison & & Validates findings across tools \\
\bottomrule
\end{tabular}
\end{center}

\subsection*{Computational Cost Comparison}

\begin{center}
\begin{tabular}{l r r r}
\toprule
\textbf{Method} & \textbf{Model 1 Time (s)} & \textbf{Model 2 Time (s)} & \textbf{Speedup vs.~Gibbs} \\
\midrule
Mean-Field VI & 0.02–0.05 & 0.05–0.15 & $128\times$ to $1,709\times$ \\
Hybrid (VI + 100 Gibbs iter.) & 0.50–1.00 & 1.50–3.00 & $13\times$ to $68\times$ \\
Gibbs (10,000 iter., baseline) & 3.50–6.40 & 20.0–102.0 & $1.0\times$ \\
Stan NUTS (estimated) & 1.00–2.00 & 5.00–15.0 & $2\times$ to $20\times$ \\
\bottomrule
\end{tabular}
\end{center}

\subsection*{Implementation Priority}

\begin{enumerate}
\item \textbf{Current (completed):} Mean-field VI and comprehensive SD ratio diagnostics (this report)
\item \textbf{Next phase:} Implement hybrid method and compare with pure Gibbs to quantify burn-in savings
\item \textbf{Parallel:} Learn Stan/NUTS and validate that under-dispersion findings hold independently
\item \textbf{Long-term:} Productionise hybrid approach with diagnostic monitoring
\end{enumerate}
